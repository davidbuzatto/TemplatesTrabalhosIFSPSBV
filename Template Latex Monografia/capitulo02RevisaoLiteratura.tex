\chapter{Revisão da Literatura}
\label{cap:02}

Texto da revisão da literatura, dividido em seções e subseções.


Este é um exemplo de como usar figuras. Referência cruzada: Figura~\ref{fig:exemplo}

\FloatBarrier
\begin{figure}[!htbp]
	\centering
	\caption{Exemplo de figura}
	%scale redimensiona a figura.
	%1 = 100% do tamanho original
	%0.20 = 20% do tamanho original
	\includegraphics[scale=0.20]{imagens/exemplo}
	\\\textbf{Fonte:} Elaborada pelo autor
	\label{fig:exemplo}
\end{figure}
\FloatBarrier


Este é um exemplo de como usar tabelas. Referência cruzada: Tabela~\ref{tab:exemplo}

\FloatBarrier
\begin{table}[!htbp]
\centering
\caption{Exemplo de tabela de 2 colunas}
	\begin{tabular}{ c | c }
		\hline
		\textbf{Coluna 1} & \textbf{Coluna 2} \\ \hline
		Dado 1a           & Dado 1b           \\ \hline
		Dado 2a           & Dado 2b           \\ \hline
		Dado 3a           & Dado 3b           \\ \hline
		Dado 4a           & Dado 4b           \\ \hline
	\end{tabular}
	\\ \vspace{0.2cm}
	\textbf{Fonte:} Elaborada pelo autor
	\label{tab:exemplo}
\end{table}
\FloatBarrier


Este é um exemplo de como usar equações. Referência cruzada: Equação~\ref{eq:exemplo}

\begin{equation}
\sum_{i=1}^{n} = \frac{n(n+1)}{2}
\label{eq:exemplo}
\end{equation}


Exemplo de inserção de lista de código fonte (não use acentos no código!):

\lstinputlisting[language=Java]{fontes/Grafo.java} 



Este é um exemplo de como inserir texto sem formatação (ambiente verbatim):

\begin{verbatim}
	Texto sem formatação, como espaçamento igual.
\end{verbatim}


Exemplo de lista de itens:

\begin{itemize}
	\item \textbf{Item 1:} texto...;
	\item \textbf{Item 2:} texto...;
	\item \textbf{Item 3:} texto...;
	\item \textbf{Item n:} texto...;
\end{itemize}


Exemplo de lista numerada:

\begin{enumerate}
	\item \textbf{Item:} texto...;
	\item \textbf{Item:} texto...;
	\item \textbf{Item:} texto...;
	\item \textbf{Item:} texto...;
\end{enumerate}


Exemplos de comandos para texto e referências:

\begin{itemize}
	\item Negrito (comando \verb|\textbf|): \textbf{texto em negrito};
	\item Itálico (comando \verb|\textit|): \textit{texto em itálico};
	\item Sublinhado (comando \verb|\underline|): \underline{texto sublinhado};
	\item Negrito e itálico (usar comandos juntos): \textbf{\textit{texto em negrito e itálico}};
	\item Ambiente matemático inline (comando \verb|$ expressão $|): $s = x^2-2x +1$;
	\item Referência normal (comando \verb|\cite|):
	\begin{itemize}
		\item \cite{Agaisse1995};
		\item \cite{Abedi2014};
		\item \cite{AgapitoTenfen2014};
	\end{itemize}
	\item Referência nome e ano (comando \verb|\citeauthorandyear|):
	\begin{itemize}
		\item \citeauthorandyear{Agaisse1995};
		\item \citeauthorandyear{Abedi2014};
		\item \citeauthorandyear{AgapitoTenfen2014};
	\end{itemize}
\end{itemize}


Exemplo 1 de referência direta:

\begin{citacao}
	Os 20 aminoácidos usualmente encontrados como resíduos em proteínas contém um grupo $\alpha$-carboxil, um grupo $\alpha$-amino e um grupo R distinto substituído no átomo de carbono $\alpha$. O átomo de carbono $\alpha$ de todos os aminoácidos, com exceção da glicina, é assimétrico e, portanto, os aminoácidos podem existir em pelo menos duas formas estereoisoméricas. Somente os estereoisômeros L, com uma configuração relacionada à configuração absoluta da molécula de referência L-gliceraldeído, são encontrados em proteínas \cite[p. 81]{Nelson2014}
\end{citacao}

Exemplo 2 de referência direta:

\begin{citacao}
	\textit{These various insecticidal proteins are synthesized during the stationary phase and accumulate in the mother cell as a crystal inclusion which can account for up to 25\% of the dry weight of the sporulated cells. The amount of crystal protein produced by a B. thuringiensis culture in laboratory conditions (about 0.5 mg of protein per ml) and the size of the crystals (24) indicate that each cell has to synthesize $10^6$ to $2 \times 10^6$ $\delta$-endotoxin molecules during the stationary phase to form a crystal} \cite[p. 1]{Agaisse1995}
\end{citacao}

Exemplo de nota de rodapé\footnote{Essa é uma nota de rodapé!}.
