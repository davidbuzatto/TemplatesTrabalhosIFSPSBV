% Desenvolvido por: Prof. Dr. David Buzatto
%
% Versão 1.6.1
% Data: 30/11/2022

\documentclass[
	12pt,
   	oneside,
   	a4paper,
   	chapter=TITLE,
   	english,
   	french,
   	spanish,
   	brazil,
]{abntex2}

\usepackage{estrutura}
\usepackage[brazilian,hyperpageref]{backref}
\usepackage[alf,abnt-emphasize=bf]{abntex2cite}

% ---
% Dados do documento
% ---

\tipotrabalho{Relatório Técnico}

\titulo{Título}
% caso não haja, comente a linha abaixo
\subtitulo{subtítulo (se houver)}

\autor{Nome Completo}
\orientador{Prof./Profa. Me./Dr./Dra. Nome Completo}
% caso não haja, comente a linha abaixo
\coorientador{Prof./Profa. Me./Dr./Dra. Nome Completo}

\curso{Nome do Curso}
\grau{Definição do grau}

%exemplos
%\curso{Bacharelado em Ciência da Computação}
%\grau{Bacharel em Ciência da Computação em Sistemas para Internet}
%\curso{Especialização em Desenvolvimento de Aplicações para Dispositivos Móveis}
%\grau{Especialista em Desenvolvimento de Aplicações para Dispositivos Móveis}

\campus{São João da Boa Vista}
\area{Área de Concentração do Trabalho}

\local{São João da Boa Vista}
\mes{MÊS}
\ano{ANO}

\instituicao{%
	Instituto Federal de Educação, Ciência e Tecnologia de São Paulo
	\par
	Câmpus \imprimircampus
}

\preambulo{\imprimirtipotrabalho\ elaborado conforme a ABNT NBR 10719:10, apresentado ao Instituto Federal de Educação, Ciência e Tecnologia de São Paulo, como parte dos requisitos para a obtenção do grau de \imprimirgrau.
\\
\\
Área de Concentração: \imprimirarea}

\setlength{\parindent}{1.3cm}
\setlength{\parskip}{0.2cm}

\makeindex

% ---------------------------------------------------------------------------------
%                                   INÍCIO DO DOCUMENTO
% ---------------------------------------------------------------------------------
\begin{document}

% Seleciona o idioma do documento
\selectlanguage{brazil}

% Retira espaço extra obsoleto entre as frases.
\frenchspacing 

\pretextual
\input{01Capa}
\newcommand{\specialcell}[2][c]{%
	\begin{tabular}[#1]{@{}c@{}}#2\end{tabular}}

\FloatBarrier
\begin{table}[!htbp]
	\centering
	\renewcommand{\arraystretch}{1.5}% Spread rows out...
	\begin{tabular}{| >{\centering}m{2in} | >{\centering}m{2in} | >{\centering\arraybackslash}m{2in} | }
		\hline
		\specialcell[t]{INSTITUTO FEDERAL DE\\EDUCAÇÃO, CIÊNCIA E\\TECNOLOGIA DE SÃO\\PAULO - CÂMPUS SÃO\\JOÃO DA BOA VISTA} & \imprimirmes & \imprimirano \\
		\hline
		\multicolumn{3}{|c|}{\specialcell{
				
				\\ \\
				
				\imprimircurso
				
				\\ \\ \\ \\ \\ \\
				
				\begin{minipage}[t]{0.9\columnwidth}%
				    \centering
				    \ABNTEXchapterfont\LARGE\imprimirtitulo\ifdef{\osubtitulo}{:}{}
                    
                    \ifdef{\osubtitulo}{\ABNTEXchapterfont\Large\imprimirsubtitulo}{}
                    
				\end{minipage}
				
				\\ \\ \\ \\
				
				}}\\
		\multicolumn{3}{|r|}{\specialcell{
				
                \begin{minipage}[t]{0.9\columnwidth}%
   				    \flushright
    				\imprimirautor \ifdef{\ocoorientador}{,}{\ e}
                    \\
                    \imprimirorientador \ifdef{\ocoorientador}{ e\\\imprimircoorientador}{}
                \end{minipage}
				
				\\ \\ \\ \\ \\ \\ \\ \\ \\ \\ 
				
			}}\\
			\hline
		\multicolumn{2}{|l|}{\specialcell{Palavras-chave:\\Palavra-chave 1. Palavra-chave 2. Palavra-chave n.}} & \pageref{LastPage} páginas\\
		\hline
	\end{tabular}
\end{table}
\FloatBarrier

% ---
% Inserir a ficha catalográfica
% ---
\input{03FichaCatalografica}

% ---
% Inserir ata de defesa
% ---
%
% Este é um exemplo da ata de defesa.
%
% A ata de defesa será fornecida pelo professor orientador após a defesa do trabalho
% e o orientado será responsável em substituir o arquivo de exemplo pelo arquivo final.
%
\begin{folhadeaprovacao}
	\includepdf{ataDefesa/exemploAtaDefesa.pdf}
\end{folhadeaprovacao}

% ---
% inserir o sumário
% ---
\pdfbookmark[0]{\contentsname}{toc}
\tableofcontents*
\cleardoublepage
% ---

\chapter*{}
\noindent{\textbf{RESUMO}}

\noindent{Neste trabalho é apresentada a formatação que deve ser utilizada nos relatórios técnicos a serem submetidos ao final dos cursos de Graduação e Pós-graduação do IFSP câmpus São João da Boa Vista. Leia com atenção este documento. O máximo de palavras para o resumo é 150 (cento e cinquenta).}

\vspace{\onelineskip}

% todas em letras minúsculas, separadas por ponto e vírgula (;)
\noindent{\textbf{Palavras-chave}: palavra-chave 1. palavra-chave 2. palavra-chave 3. palavra-chave n.}

\textual
\input{capitulo01Introducao}
\input{capitulo02ConsideracoesGerais}
\chapter{Metodologia}
\label{cap:03}

Texto da metodologia.
\chapter{Análise dos Resultados}
\label{cap:04}

Relatar os resultados obtidos a partir dos experimentos e dos estudos realizados. 


\section{Resultados/Impactos}

Resultados


\section{Orçamento}

Orçamento, caso exista


\section{Cronograma do trabalho}

Cronograma

\input{capitulo05ConclusoesRecomendacoes}


% ----------------------------------------------------------
% Referências bibliográficas
% ----------------------------------------------------------
\postextual
\bibliography{referencias}

\end{document}
