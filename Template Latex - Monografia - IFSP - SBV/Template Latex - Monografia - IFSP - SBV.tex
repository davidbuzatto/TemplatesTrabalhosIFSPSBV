% Desenvolvido por: Prof. Dr. David Buzatto
%
% Versão 1.6.1
% Data: 30/11/2022

\documentclass[
	12pt,
	openright,
   	twoside,
   	a4paper,
   	chapter=TITLE,
   	english,
   	french,
   	spanish,
   	brazil
]{abntex2}

\usepackage{estrutura}
\usepackage[brazilian,hyperpageref]{backref}
\usepackage[alf,abnt-emphasize=bf]{abntex2cite}

% ---
% Dados do documento
% ---

\tipotrabalho{Trabalho de Conclusão de Curso}

\titulo{Título}
% caso não haja, comente a linha abaixo
\subtitulo{subtítulo (se houver)}

\autor{Nome Completo}
\orientador{Prof./Profa. Me./Dr./Dra. Nome Completo}
% caso não haja, comente a linha abaixo
\coorientador{Prof./Profa. Me./Dr./Dra. Nome Completo}

\curso{Nome do Curso}
\grau{Definição do grau}

%exemplos
%\curso{Bacharelado em Ciência da Computação}
%\grau{Bacharel em Ciência da Computação em Sistemas para Internet}
%\curso{Especialização em Desenvolvimento de Aplicações para Dispositivos Móveis}
%\grau{Especialista em Desenvolvimento de Aplicações para Dispositivos Móveis}

\campus{São João da Boa Vista}
\area{Área de Concentração do Trabalho}

\local{São João da Boa Vista}
\mes{MÊS}
\ano{ANO}

\instituicao{%
	Instituto Federal de Educação, Ciência e Tecnologia de São Paulo
	\par
	Câmpus \imprimircampus
}

\preambulo{\imprimirtipotrabalho\ apresentado ao Instituto Federal de Educação, Ciência e Tecnologia de São Paulo, como parte dos requisitos para a obtenção do grau de \imprimirgrau.
\\
\\
Área de Concentração: \imprimirarea}

\setlength{\parindent}{1.3cm}
\setlength{\parskip}{0.2cm}

\makeindex

% ---------------------------------------------------------------------------------
%                                   INÍCIO DO DOCUMENTO
% ---------------------------------------------------------------------------------
\begin{document}
	
% Seleciona o idioma do documento (conforme pacotes do babel)
%\selectlanguage{english}
\selectlanguage{brazil}

% Retira espaço extra obsoleto entre as frases.
\frenchspacing 


% ---------------------------------------------------------------------------------
%                                   ELEMENTOS PRÉ-TEXTUAIS
% ---------------------------------------------------------------------------------
\pretextual

% ---
% Capa
% ---
%\imprimircapa
% capa personalizada

\begin{center}
	
	%\center
	\ABNTEXchapterfont\Large\textsc{\imprimirinstituicao}
	\vspace{3.5cm}
	
    \ABNTEXchapterfont\Large\textsc{\imprimirautor}
	\vspace{3.5cm}
	
    \ABNTEXchapterfont\LARGE\textsc{\imprimirtitulo\ifdef{\osubtitulo}{:}{}}
    
    \ifdef{\osubtitulo}{\ABNTEXchapterfont\Large\imprimirsubtitulo}{}
	\vfill
	
	\Large\textsc{\imprimirlocal}
	
	\Large\textsc{\imprimirano}
	
	\vspace*{2cm}
	
\end{center}

\cleardoublepage


% ---

% ---
% Folha de rosto
% (o * indica que haverá a ficha bibliográfica)
% ---
%\imprimirfolhaderosto*
\begin{center}
   	
   	\ABNTEXchapterfont\Large\textsc{\imprimirautor}
   	\vspace{2.5cm}
   	
    \ABNTEXchapterfont\LARGE\imprimirtitulo\ifdef{\osubtitulo}{:}{}
                           
    \ifdef{\osubtitulo}{\ABNTEXchapterfont\Large\imprimirsubtitulo}{}
   	\vspace{2.5cm}
   	   	
   	\hspace{.4\textwidth}
   	\begin{minipage}{.5\textwidth}
   		\SingleSpacing
   		\large\imprimirpreambulo
   		
   		\vspace{\onelineskip}
   		
   		Orientador: \imprimirorientador
   		
        \ifdef{\ocoorientador}{
     		\vspace{\onelineskip}
   		
    		Coorientador: \imprimircoorientador
        }{}
   		
   	\end{minipage}%
    \vfill
   	
   	\Large\textsc{\imprimirlocal}
   	
   	\Large\textsc{\imprimirano}
   	
   	\vspace*{2cm}
   	
\end{center}
% ---

% ---
% Inserir a ficha catalográfica
% ---
% Isto é um exemplo de Ficha Catalográfica, ou ``Dados internacionais de
% catalogação-na-publicação''. Você pode utilizar este modelo como referência. 
% Porém, provavelmente a biblioteca da sua universidade lhe fornecerá um PDF
% com a ficha catalográfica definitiva após a defesa do trabalho. Quando estiver
% com o documento, salve-o como PDF no diretório do seu projeto e substitua todo
% o conteúdo de implementação deste arquivo pelo comando abaixo:
%
% \begin{fichacatalografica}
%     \includepdf{fig_ficha_catalografica.pdf}
% \end{fichacatalografica}

\begin{fichacatalografica}
	
	Folha destinada à inclusão da Catalogação na Fonte - Ficha Catalográfica (a ser solicitada à Biblioteca IFSP – Câmpus São João da Boa Vista e posteriormente impressa no verso da Folha de Rosto (folha anterior).
	
	\vspace{3cm}
	
	\begin{center}
		Catalogação na Fonte preparada pela Biblioteca Comunitária “Wolgran Junqueira Ferreira” do IFSP – Câmpus São João da Boa Vista
	\end{center}
	
	
	\sffamily
	\vspace*{\fill}					% Posição vertical
	\begin{center}					% Minipage Centralizado
		\fbox{\begin{minipage}[c][8cm]{13.5cm}		% Largura
			\small
			\imprimirautor
			%Sobrenome, Nome do autor
			
			\hspace{0.5cm} \imprimirtitulo  / \imprimirautor. --
			\imprimirlocal, \imprimirano-
			
			\hspace{0.5cm} \pageref{LastPage} p. : il. (algumas color.) ; 30 cm.\\
			
			\hspace{0.5cm} Orientador ~Prof. Dr. \imprimirorientador\\
			
			\hspace{0.5cm}
			\parbox[t]{\textwidth}{\imprimirtipotrabalho~--~\imprimirinstituicao,
				\imprimirano.}\\
			
			\hspace{0.5cm}
			1. Palavra-chave 1.
			2. Palavra-chave 2.
			3. Palavra-chave 3.
			I. Orientador.
			II. Instituto Federal de Educação, Ciência e Tecnologia de São Paulo.
			III. Título 			
		\end{minipage}}
	\end{center}
\end{fichacatalografica}

% ---
% Inserir ata de defesa
% ---
\input{pre04AtaDefesa}

% ---
% Dedicatória
% ---
\begin{dedicatoria}
	\vspace*{\fill}
	\centering
	\noindent
	\textit{Dedicatória (Opcional). Não digite a palavra Dedicatória. Texto no qual o autor do trabalho oferece homenagem ou dedica o seu trabalho a alguém (não usar ponto final)}
    \vspace*{\fill}
\end{dedicatoria}

% ---
% Agradecimentos
% ---
\input{pre06Agradecimentos}

% ---
% Epígrafe
% ---
\input{pre07Epigrafe}

% ---
% Resumos
% ---
\noindent{SOBRENOME, Prenome. \textbf{Título do trabalho de TCC colocado em negrito:} subtítulo (se houver). Ano da defesa. Tipo de documento (Grau e vinculação acadêmica) – Instituição, Local. Ano da entrega.}

\noindent{Exemplo: BUZATTO, David. \textbf{Especificidade de Proteínas Cry de \textit{Bacillus thuringiensis} Baseada nas Características Conformacionais de Suas Estruturas Terciárias}. 2017. Tese (Doutorado em Biotecnologia) – Universidade de Ribeirão Preto, Ribeirão Preto. 2017.}


\setlength{\absparsep}{18pt} % ajusta o espaçamento dos parágrafos do resumo
\begin{resumo}
	
	Elemento obrigatório, constituído de uma sequência de frases concisas e objetivas, fornecendo uma visão rápida e clara do conteúdo do estudo. O texto deverá conter entre 150 a 250 palavras e ser antecedido pela referência do estudo. Também, não deve conter citações e deverá ressaltar o objetivo, o método, os resultados e as conclusões. O resumo deve ser redigido em parágrafo único, seguido das palavras representativas do conteúdo do estudo, isto é, palavras-chave, em número de três a cinco, separadas entre si por ponto e finalizadas também por ponto. Usar o verbo na terceira pessoa do singular, com linguagem impessoal (pronome SE), bem como fazer uso, preferencialmente, da voz ativa.
	
	\vspace{\onelineskip}
	
	\textbf{Palavras-chave}: Palavra-chave 1. Palavra-chave 2. Palavra-chave 3. Palavra-chave n.
	
\end{resumo}
\noindent{SOBRENOME, Prenome. \textbf{Título do trabalho de TCC colocado em negrito:} subtítulo (se houver). Ano da defesa. Tipo de documento (Grau e vinculação acadêmica) – Instituição, Local. Ano da entrega.}

% resumo em inglês
\begin{resumo}[Abstract]
	\begin{otherlanguage*}{english}
		
		Elemento obrigatório. É a versão do resumo em português para o idioma de divulgação internacional. Deve ser antecedido pela referência do estudo.
		
		\vspace{\onelineskip}
		 
		\textbf{Keywords}: Keyword 1. Keyword 2. Keyword 3. Keyword n.
	\end{otherlanguage*}
\end{resumo} 

% ---
% inserir lista de ilustrações
% ---
\pdfbookmark[0]{\listfigurename}{lof}
\listoffigures*
\cleardoublepage
% ---

% ---
% inserir lista de quadros
% ---
\pdfbookmark[0]{\listofquadrosname}{loq}
\listofquadros*
\cleardoublepage
% ---

% ---
% inserir lista de tabelas
% ---
\pdfbookmark[0]{\listtablename}{lot}
\listoftables*
\cleardoublepage
% ---

% ---
% inserir lista de abreviaturas e siglas
% ---
\input{pre10ListaSiglas}
% ---

% ---
% inserir lista de símbolos
% ---
\input{pre11ListaSimbolos}
% ---

% ---
% inserir o sumário
% ---
\pdfbookmark[0]{\contentsname}{toc}
\tableofcontents*
\cleardoublepage
% ---






% ---------------------------------------------------------------------------------
%                                  ELEMENTOS TEXTUAIS
% ---------------------------------------------------------------------------------
\textual

\input{capitulo01Introducao}
\chapter{Revisão da Literatura}
\label{cap:02}

Texto da revisão da literatura, dividido em seções e subseções.


Este é um exemplo de como usar figuras. Referência cruzada: Figura~\ref{fig:exemplo}

\FloatBarrier
\begin{figure}[!htbp]
	\centering
	\caption{Exemplo de figura}
	%scale redimensiona a figura.
	%1 = 100% do tamanho original
	%0.20 = 20% do tamanho original
	\includegraphics[scale=0.20]{imagens/exemplo}
	\\\textbf{Fonte:} Elaborada pelo autor
	\label{fig:exemplo}
\end{figure}
\FloatBarrier


Este é um exemplo de como usar tabelas. Referência cruzada: Tabela~\ref{tab:exemplo}

\FloatBarrier
\begin{table}[!htbp]
\centering
\caption{Exemplo de tabela de 2 colunas}
	\begin{tabular}{ c | c }
		\hline
		\textbf{Coluna 1} & \textbf{Coluna 2} \\ \hline
		Dado 1a           & Dado 1b           \\ \hline
		Dado 2a           & Dado 2b           \\ \hline
		Dado 3a           & Dado 3b           \\ \hline
		Dado 4a           & Dado 4b           \\ \hline
	\end{tabular}
	\\ \vspace{0.2cm}
	\textbf{Fonte:} Elaborada pelo autor
	\label{tab:exemplo}
\end{table}
\FloatBarrier


Este é um exemplo de como usar equações. Referência cruzada: Equação~\ref{eq:exemplo}

\begin{equation}
\sum_{i=1}^{n} = \frac{n(n+1)}{2}
\label{eq:exemplo}
\end{equation}


Exemplo de inserção de lista de código fonte (não use acentos no código!):

\lstinputlisting[language=Java]{fontes/Grafo.java} 



Este é um exemplo de como inserir texto sem formatação (ambiente verbatim):

\begin{verbatim}
	Texto sem formatação, como espaçamento igual.
\end{verbatim}


Exemplo de lista de itens:

\begin{itemize}
	\item \textbf{Item 1:} texto...;
	\item \textbf{Item 2:} texto...;
	\item \textbf{Item 3:} texto...;
	\item \textbf{Item n:} texto...;
\end{itemize}


Exemplo de lista numerada:

\begin{enumerate}
	\item \textbf{Item:} texto...;
	\item \textbf{Item:} texto...;
	\item \textbf{Item:} texto...;
	\item \textbf{Item:} texto...;
\end{enumerate}


Exemplos de comandos para texto e referências:

\begin{itemize}
	\item Negrito (comando \verb|\textbf|): \textbf{texto em negrito};
	\item Itálico (comando \verb|\textit|): \textit{texto em itálico};
	\item Sublinhado (comando \verb|\underline|): \underline{texto sublinhado};
	\item Negrito e itálico (usar comandos juntos): \textbf{\textit{texto em negrito e itálico}};
	\item Ambiente matemático inline (comando \verb|$ expressão $|): $s = x^2-2x +1$;
	\item Referência normal (comando \verb|\cite|):
	\begin{itemize}
		\item \cite{Agaisse1995};
		\item \cite{Abedi2014};
		\item \cite{AgapitoTenfen2014};
	\end{itemize}
	\item Referência nome e ano (comando \verb|\citeauthorandyear|):
	\begin{itemize}
		\item \citeauthorandyear{Agaisse1995};
		\item \citeauthorandyear{Abedi2014};
		\item \citeauthorandyear{AgapitoTenfen2014};
	\end{itemize}
\end{itemize}


Exemplo 1 de referência direta:

\begin{citacao}
	Os 20 aminoácidos usualmente encontrados como resíduos em proteínas contém um grupo $\alpha$-carboxil, um grupo $\alpha$-amino e um grupo R distinto substituído no átomo de carbono $\alpha$. O átomo de carbono $\alpha$ de todos os aminoácidos, com exceção da glicina, é assimétrico e, portanto, os aminoácidos podem existir em pelo menos duas formas estereoisoméricas. Somente os estereoisômeros L, com uma configuração relacionada à configuração absoluta da molécula de referência L-gliceraldeído, são encontrados em proteínas \cite[p. 81]{Nelson2014}
\end{citacao}

Exemplo 2 de referência direta:

\begin{citacao}
	\textit{These various insecticidal proteins are synthesized during the stationary phase and accumulate in the mother cell as a crystal inclusion which can account for up to 25\% of the dry weight of the sporulated cells. The amount of crystal protein produced by a B. thuringiensis culture in laboratory conditions (about 0.5 mg of protein per ml) and the size of the crystals (24) indicate that each cell has to synthesize $10^6$ to $2 \times 10^6$ $\delta$-endotoxin molecules during the stationary phase to form a crystal} \cite[p. 1]{Agaisse1995}
\end{citacao}

Exemplo de nota de rodapé\footnote{Essa é uma nota de rodapé!}.

\chapter{Metodologia}
\label{cap:03}

Texto da metodologia.
\chapter{Resultados e Discussão}
\label{cap:04}

Texto dos resultados.
\chapter{Conclusões/Conclusões Parciais}
\label{cap:05}

Texto das conclusões (as conclusões parciais são para a graduação na qualificação).

\chapter{Cronograma}
\label{cap:06}

Cronograma (para a graduação na qualificação)






% ---------------------------------------------------------------------------------
%                                 ELEMENTOS PÓS-TEXTUAIS
% ---------------------------------------------------------------------------------
\postextual


% ----------------------------------------------------------
% Referências bibliográficas
% ----------------------------------------------------------
\bibliography{referencias}


% ----------------------------------------------------------
% Glossário
% ----------------------------------------------------------
%
% Consulte o manual da classe abntex2 para orientações sobre o glossário.
%
%\glossary


% ----------------------------------------------------------
% Apêndices
% ----------------------------------------------------------
% Texto ou documento elaborado pelo autor, a fim de complementar sua argumentação, sem prejuízo da unidade nuclear do trabalho.

% ---
% Inicia os apêndices
% ---
\begin{apendicesenv}
	
	% Imprime uma página indicando o início dos apêndices
	\partapendices
	
	% ----------------------------------------------------------
	\chapter{Título do Apêndice A}
	% ----------------------------------------------------------
	
	Texto do Apêndice A.
	
	
	
	% ----------------------------------------------------------
	\chapter{Título do Apêndice B}
	% ----------------------------------------------------------
	
	Texto do Apêndice B.
	
	
	
	% ----------------------------------------------------------
	\chapter{Título do Apêndice C}
	% ----------------------------------------------------------
	
	Texto do Apêndice C.
	
	
	
	% ----------------------------------------------------------
	\chapter{Título do Apêndice D}
	% ----------------------------------------------------------
	
	Texto do Apêndice D.
	
	
	
	% ----------------------------------------------------------
	\chapter{Título do Apêndice E}
	% ----------------------------------------------------------
	
	Texto do Apêndice E.
	
	
	
\end{apendicesenv}
% ---


% ----------------------------------------------------------
% Anexos
% ----------------------------------------------------------
% Texto ou documento não elaborado pelo autor, que serve de fundamentação, comprovação e ilustração.

% ---
% Inicia os anexos
% ---
\begin{anexosenv}
	
	% Imprime uma página indicando o início dos anexos
	\partanexos
	
	% ----------------------------------------------------------
	\chapter{Título do Anexo A}
	% ----------------------------------------------------------
	
	Texto do Anexo A.
	
	
	
	% ----------------------------------------------------------
	\chapter{Título do Anexo B}
	% ----------------------------------------------------------
	
	Texto do Anexo B.
	
	
	
	% ----------------------------------------------------------
	\chapter{Título do Anexo C}
	% ----------------------------------------------------------
	
	Texto do Anexo C.
	
	
	
	% ----------------------------------------------------------
	\chapter{Título do Anexo D}
	% ----------------------------------------------------------
	
	Texto do Anexo D.
	
	
	
	% ----------------------------------------------------------
	\chapter{Título do Anexo E}
	% ----------------------------------------------------------
	
	Texto do Anexo E.
	
	
	
\end{anexosenv}


%---------------------------------------------------------------------
% ÍNDICE REMISSIVO
%---------------------------------------------------------------------

%\phantompart

%\printindex


\end{document}
